\documentclass[12pt, twoside]{article}
\usepackage[utf8]{inputenc}
\usepackage[english,russian]{babel}
\newcommand{\hdir}{.}

\usepackage{graphicx}
\usepackage{caption}
\usepackage{amssymb}
\usepackage{amsmath}
\usepackage{mathrsfs}
\usepackage{euscript}
\usepackage{upgreek}
\usepackage{array}
\usepackage{theorem}
\usepackage{graphicx}
\usepackage{subfig}
\usepackage{caption}
\usepackage{color}
\usepackage{url}

\usepackage[left=2cm,right=2cm,top=3cm,bottom=2cm,bindingoffset=0cm]{geometry}

\usepackage{fancyhdr}
\pagestyle{fancy}
\fancyhead{}
\fancyhead[LE,RO]{\thepage} 
\fancyhead[CO,CE]{Лекция 9}
\fancyhead[LO,LE]{Грабовой Андрей}

\begin{document} 

\begin{center}
{\LARGE\bf
Динамическое программирование
}
\end{center}

\section{Задача Минимальный путь в таблице\cite{Minpath}}

\paragraph{Условие задачи.} В прямоугольной таблице $N\times M$ (в каждой клетке которой записано некоторое число) в начале игрок находится в левой верхней клетке. За один ход ему разрешается перемещаться в соседнюю клетку либо вправо, либо вниз (влево и вверх перемещаться запрещено). При проходе через клетку с игрока берут столько у.е., какое число записано в этой клетке (деньги берут также за первую и последнюю клетки его пути).

Требуется найти минимальную сумму у.е., заплатив которую игрок может попасть в правый нижний угол.
\paragraph{Входные данные.} На входе задано два числа $N$ и $M$ - размеры таблицы $(1 \leq N \leq 20, 1 \leq M \leq 20)$. Затем идет $N$ строк по $M$ чисел в каждой - размеры штрафов в у.е. за прохождение через соответствующие клетки (числа от $0$ до $100$).
\paragraph{Выходные данные.} На выходе выведите минимальную сумму, потратив которую можно попасть в правый нижний угол.
\begin{thebibliography}{99}
	\bibitem{Minpath}
	\textit{Минимальный путь в таблице} {http://acmp.ru/index.asp?main=task\&id\_task=120}
\end{thebibliography}



\end{document} 