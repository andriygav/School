\documentclass[12pt, twoside]{article}
\usepackage[utf8]{inputenc}
\usepackage[english,russian]{babel}
\newcommand{\hdir}{.}

\usepackage{graphicx}
\usepackage{caption}
\usepackage{amssymb}
\usepackage{amsmath}
\usepackage{mathrsfs}
\usepackage{euscript}
\usepackage{upgreek}
\usepackage{array}
\usepackage{theorem}
\usepackage{graphicx}
\usepackage{subfig}
\usepackage{caption}
\usepackage{color}
\usepackage{url}

\usepackage[left=2cm,right=2cm,top=3cm,bottom=2cm,bindingoffset=0cm]{geometry}

\usepackage{fancyhdr}
\pagestyle{fancy}
\fancyhead{}
\fancyhead[LE,RO]{\thepage} 
\fancyhead[CO,CE]{Final Test}
\fancyhead[LO,LE]{Ф.И.О.}

\begin{document} 

\paragraph{1.} Что такое матрица?

\paragraph{2.} Условия на размеры матриц $\textbf{A}_{n, m}$ и  $\textbf{B}_{r, p}$ для чтобы операция матричного произведение $\textbf{C} = \textbf{A}\times\textbf{B}$ была коректна.

\paragraph{3.} Найти произведение двух матриц:
$$\textbf{C} = \textbf{A}\times\textbf{B}, 
\quad \textbf{A} = 
\begin{bmatrix}
1&2\\
3&4
\end{bmatrix},
\quad \textbf{B} = 
\begin{bmatrix}
5&6\\
7&8
\end{bmatrix},
$$

\paragraph{4.} Что такое задача регрессии?

\paragraph{5.} Что такое задача классификации?

\paragraph{6.} Определение линейной регрессии.

\paragraph{7.} Определение логистической регрессии (линейной классификации).

\paragraph{8.} Функция ошибки для линейной регрессии и логистической регрессии

\paragraph{9.} Формулировка задачи машинного обучения, как поиск оптимального $\textbf{w}$.

\paragraph{10.} Найти минимум $\left(\text{производная по всем компонентам ноль}\right)$ функции $f$:
$$f\left(\textbf{x}\right) = x_1^2 + 2x_2^2+3x_3^2+4x_4^2.$$

\paragraph{11.} Выписать итеративную формулу градиентного спуска для произвольной функции $f\left(\textbf{x}\right).$

\paragraph{12.} Выписать итеративную формулу градиентного спуска для функции $f\left(\textbf{x}\right) = \textbf{x}^{\mathsf{T}}\textbf{a}.$

\paragraph{13.} При помощи метода <<Автоматического дифференцирования назад>> найти производную функции:
$$f\left(\textbf{x}\right) = \frac{x_1x_3sin\left(x_2\right) + \exp\left(x_1x_3\right)}{x_2}$$

\paragraph{14.} Нарисовать структуру нейронной сети <<Многослойный перцептрон (Fully Connected Neural Network)>>.

\paragraph{15.} Какие Вы знаете точечные преобразования изображений?

\paragraph{16.} Какие Вы знаете пространственные преобразования изображений?

\paragraph{17.} Что такое свертка?

\paragraph{18.} Что такое max-poling?

\paragraph{19.} Структура нейронной сети <<LSTM>>.

\paragraph{20.} Что такое <<seq2seq>> модель?

\paragraph{21.} Опишите задачу <<image2caption>>.








\end{document} 