\section{Самостоятельная работа. Опиисание выборки.}

\href{https://github.com/andriygav/School/blob/master/2018/AD/Lecture/Lecture6.ipynb}{Ссылка на ноутбук}

\subsection{Wine\cite{Wine}}
Эти данные являются результатом химического анализа вин, выращенных в одном и том же регионе в Италии, но полученных из трех разных сортов. Анализ определил количество 13 составляющих, найденных в каждом из трех типов вин.

\begin{table}[h]
\begin{center}
\caption{Описание выборки Wine}
\begin{tabular}{|c|c|c|c|}
\hline
Алкоголь&Малиновая кислота&Зола&Алкалиния золы\\
\hline
Магний&Всего фенолов&Флаваноиды&Нефлаваноидные фенолы\\
\hline
Проантоцианы&Интенсивность цвета&Оттенок&OD280 / OD315 разведенных вин\\
\hline
Пролин&\bf Тип винограда&&\\
\hline
\end{tabular}
\end{center}
\end{table}

\subsection{Iris\cite{Iris}}
Это, пожалуй, самая известная база данных, которая может быть найдена в литературе по распознаванию образов. Статья Фишера является классикой в этой области и часто упоминается по сей день. (Например, Duda \& Hart). Набор данных содержит 3 класса по 50 экземпляров каждый, где каждый класс относится к типу ириса.

\begin{table}[h]
\begin{center}
\caption{Описание выборки Iris}
\begin{tabular}{|c|c|c|c|}
\hline
Длина чашелистника &Ширина чашелистник&Длина лепестка&Ширина лепестка\\
\hline
\bf Тип ириса&&&\\
\hline
\end{tabular}
\end{center}
\end{table}

\subsection{Self-Noise\cite{SelfNoise}}
Набор данных НАСА, полученный из серии аэродинамических и акустических испытаний двух и трехмерных участков лезвия аэродинамического профиля, выполненных в безэховой аэродинамической трубе. Набор данных NASA содержит аэродинамические профили NACA 0012 различного размера на различных скоростях и углах атаки в аэродинамической трубе. Пролет аэродинамического профиля и положение наблюдателя были одинаковыми во всех экспериментах.

\begin{table}[h]
\begin{center}
\caption{Описание выборки Self-Noise}
\begin{tabular}{|c|c|c|c|}
\hline
Частота &Угол атаки&Длина волны&Скорость потока\\
\hline
Толщина смещения&\bf Уровень звукового давления&&\\
\hline
\end{tabular}
\end{center}
\end{table}

\subsection{Yacht Hydrodynamics\cite{Yacht}}
Прогнозирование остаточного сопротивления парусных яхт на начальном этапе проектирования имеет большое значение для оценки эффективности судна и оценки требуемой пропульсивной мощности. Основные входы включают основные размеры корпуса и скорость лодки. Набор данных Delft содержит 308 полномасштабных экспериментов, которые были выполнены в Лаборатории гидромеханики Делфта для этой цели.

\begin{itemize}
\item Продольное положение центра плавучести
\item Призматический коэффициент
\item Отношение длины и смещения
\item Соотношение лучей
\item Отношение длины к лучу
\item Число Фруда
\item {\bf Сопротивление резистивности}
\end{itemize}