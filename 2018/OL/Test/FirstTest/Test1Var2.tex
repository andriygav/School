\documentclass[12pt, twoside]{article}
\usepackage[utf8]{inputenc}
\usepackage[english,russian]{babel}
\newcommand{\hdir}{.}

\usepackage{graphicx}
\usepackage{caption}
\usepackage{amssymb}
\usepackage{amsmath}
\usepackage{mathrsfs}
\usepackage{euscript}
\usepackage{upgreek}
\usepackage{array}
\usepackage{theorem}
\usepackage{graphicx}
\usepackage{subfig}
\usepackage{caption}
\usepackage{color}
\usepackage{url}

\usepackage{verbatim}

\usepackage[left=2cm,right=2cm,top=3cm,bottom=2cm,bindingoffset=0cm]{geometry}

\usepackage{fancyhdr}
\pagestyle{fancy}
\fancyhead{}
\fancyhead[LE,RO]{\thepage} 
\fancyhead[CO,CE]{Контрольная работа 1-2}
\fancyhead[LO,LE]{ФИО }

\begin{document} 
\section{Задача}
Для каждого выделения памяти в программе опишите где именно оно произошло и сколько памяти было выделено.

\begin{verbatim}
#include <iostream>
#include <string.h>

const int number = 12;

int function(const char* str){
    printf("%s\n", str);
    if(strlen(str) > 0){
        function(str+1);
    }
    return 0;
}

int main(){
    char a[number];
    // вводиться Hello World
    scanf("%s", a);
    function(a);
    return 0;
}
\end{verbatim}
\section{Задача}
Найти время работы следующей функции в терминах $T(n, m)$.
\begin{verbatim}
int function(int n, int m){
    return function((n+m)/2, (n+m)/4)
}
\end{verbatim}
\section{Задача}
Доказать, что алгоритм сортировки пузырьком асимптотически работает $O(n^2)$.

\newpage
\section{Задача}
Допишите классы так, чтобы данный код компилировался и на выходе было то, что требуется. Писать код можно только там где указано. Подсказка: длина входных данных ограничена --- можно воспользоваться просто массивом в качестве хранения данных.
\begin{verbatim}
#include <iostream>
#include <stdio.h>
#include <stdlib.h>

class stack{
public:
    st();
    void push(int data);
    int pop();
private:
    //YOUR CODE HERE
};

stack::st(){
//YOUR CODE HERE
}

void stack::push(int val){
//YOUR CODE HERE
}

int stack::pop(){
//YOUR CODE HERE
}

int main(){
    stack st;
    for(int i = 0; i < 5; i++){
    	st.push(i);
    }
    for(int i; i < 5; i++){
    	printf("%d\n", st.pop());
    }
    //На экране должны появиться числа от 5 до 1 
}
\end{verbatim}
\end{document} 
